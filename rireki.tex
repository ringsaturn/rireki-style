%
% Copyright (c) 1996, 2004, 2006, 2009, 2014, 2016, 2019, 2023
% Tama Communications Corporation. All rights reserved.
%
% Redistribution and use in source and binary forms, with or without
% modification, are permitted provided that the following conditions
% are met:
% 1. Redistributions of source code must retain the above copyright
%    notice, this list of conditions and the following disclaimer.
% 2. Redistributions in binary form must reproduce the above copyright
%    notice, this list of conditions and the following disclaimer in the
%    documentation and/or other materials provided with the distribution.
%
% THIS SOFTWARE IS PROVIDED BY THE AUTHOR AND CONTRIBUTORS ``AS IS'' AND
% ANY EXPRESS OR IMPLIED WARRANTIES, INCLUDING, BUT NOT LIMITED TO, THE
% IMPLIED WARRANTIES OF MERCHANTABILITY AND FITNESS FOR A PARTICULAR PURPOSE
% ARE DISCLAIMED.  IN NO EVENT SHALL THE AUTHOR OR CONTRIBUTORS BE LIABLE
% FOR ANY DIRECT, INDIRECT, INCIDENTAL, SPECIAL, EXEMPLARY, OR CONSEQUENTIAL
% DAMAGES (INCLUDING, BUT NOT LIMITED TO, PROCUREMENT OF SUBSTITUTE GOODS
% OR SERVICES; LOSS OF USE, DATA, OR PROFITS; OR BUSINESS INTERRUPTION)
% HOWEVER CAUSED AND ON ANY THEORY OF LIABILITY, WHETHER IN CONTRACT, STRICT
% LIABILITY, OR TORT (INCLUDING NEGLIGENCE OR OTHERWISE) ARISING IN ANY WAY
% OUT OF THE USE OF THIS SOFTWARE, EVEN IF ADVISED OF THE POSSIBILITY OF
% SUCH DAMAGE.
%
%	rireki.tex	Version 3.1
%
%	URL: https://www.tamacom.com/rireki-j.html
%
%----------------------------------------------------------------------------
%
% 多摩通信社からのお知らせ
%
% 「履歴書猿人」(https://www.tamacom.com/engine/rireki2/) という履歴書作成のための
% Webアプリケーションを公開しております。TeX の知識なしにTeX の履歴書が作成でき、
% 様々なカスタマイズも可能です。こちらもご利用いただけると幸いです。
%
%----------------------------------------------------------------------------
% XeLaTeX 移行: lualatex / uplatex 分岐を削除し XeLaTeX 用クラスに統一
% bxjsarticle + 条件付きフォント選択 (resume.cls 方式)
\documentclass[xelatex,ja=standard,b5j]{bxjsarticle}
\usepackage{graphicx}
\usepackage{fontspec}
\usepackage{xeCJK}
% 日本語 Serif フォント選択優先順位: Noto Serif JP -> Hiragino Mincho ProN -> HaranoAji Mincho
\IfFontExistsTF{Noto Serif JP}{\setCJKmainfont{Noto Serif JP}}{%
	\IfFontExistsTF{Hiragino Mincho ProN}{\setCJKmainfont{Hiragino Mincho ProN}}{\setCJKmainfont{HaranoAjiMincho-Regular.otf}}%
}
% 日本語 Sans フォント選択優先順位: Noto Sans JP -> Hiragino Sans W3 -> HaranoAji Gothic
\IfFontExistsTF{Noto Sans JP}{\setCJKsansfont{Noto Sans JP}}{%
	\IfFontExistsTF{Hiragino Sans W3}{\setCJKsansfont{Hiragino Sans W3}}{\setCJKsansfont{HaranoAjiGothic-Medium.otf}}%
}
% 等幅 (任意) Noto Sans Mono CJK JP があれば設定
\IfFontExistsTF{Noto Sans Mono CJK JP}{\setCJKmonofont{Noto Sans Mono CJK JP}}{}
% 旧 pLaTeX コマンド互換 (\gt, \gtfamily)
\providecommand{\gt}{\textsf}
\providecommand{\gtfamily}{\sffamily}
\usepackage{rireki}
%
% オプション
%
% 下記のオプションが利用可能です。
%
\空行挿入		% 学歴と職歴の間に空行を挿入します
%\性別欄なし		% 性別欄を削除します
%\写真欄なし		% 写真欄を削除します
\begin{document}
%
% ID情報
%
\姓{\LARGE 履歴}
\名{\LARGE 一朗}
\姓読み{りれき}
\名読み{いちろう}
\性別{男}					% 男|女
\誕生日{平成元年2月3日}
\現在日付{令和元年5月14日}
\年齢{35}
%
% 顔写真
%
% 画像ファイルにはEPS フォーマット・縦横比4:3 のものをご使用ください。
% 縦を4cm に調整し、縦横比を変更せずに印刷します。
% 次のように指定します。
% \顔写真{photo.jpg}
%
\顔写真{}
%
% 現住所
%
\現住所郵便番号{123-4567}
\現住所{◯◯市◯◯町 1--2--3}
\現住所読み{まるまるし まるまるちょう}
\現住所市外局番{0123}
\現住所電話番号{45-6789}
\現住所呼び出し{〇〇 方}
%
% 連絡先
%
\連絡先郵便番号{}
\連絡先{\tt taro@network.or.jp}
\連絡先読み{}
\連絡先市外局番{}
\連絡先電話番号{1234-56-7890}
\連絡先呼び出し{}
%
% 学歴、職歴
%
% 学歴、職歴を年月順に列挙してください。合計20個まで記入出来ます。
% 20個を超える部分は印刷されませんので、ご注意ください。
% 印刷順は、学歴=>職歴の順になります。
%
\学歴{平成1}{4}{〇〇市立〇〇高等学校 入学}      % {年}{月}{内容}
\学歴{平成2}{3}{〇〇市立〇〇高等学校 卒業}
\学歴{平成3}{4}{〇〇大学 入学}
\学歴{平成4}{3}{〇〇大学 卒業}
\学歴{平成5}{4}{〇〇大学大学院 入学}
\学歴{平成6}{3}{〇〇大学大学院 修了}
\学歴{平成7}{4}{専門学校〇〇 入学}
\学歴{平成8}{3}{専門学校〇〇 卒業}
\職歴{平成9}{4}{株式会社〇〇 入社}
\職歴{平成10}{9}{株式会社〇〇 退職}
\職歴{平成11}{10}{株式会社〇〇 入社}
\職歴{平成12}{10}{株式会社〇〇 退職}
\職歴{平成13}{10}{有限会社〇〇 入社}
\職歴{平成14}{8}{有限会社〇〇 退職}
\学歴{平成15}{9}{〇〇〇大学 入学(海外留学)}
\学歴{平成16}{9}{〇〇〇大学 中退}
\職歴{平成17}{10}{株式会社〇〇 入社}
\職歴{平成18}{10}{株式会社〇〇 退職}
\職歴{平成18}{10}{現在無職}
%
% 資格
%
% 資格を取得年月順に列挙してください。9つまで記入できます。
% 9つを超える部分は印刷されませんので、ご注意ください。
%
\資格{平成1}{4}{普通自動車一種免許}            % {取得年}{取得月}{資格}
\資格{平成2}{9}{自動二輪免許}
\資格{平成3}{4}{第二種情報処理技術者}
\資格{平成4}{4}{第一種情報処理技術者}
\資格{平成5}{3}{宅地取り引き主任者}
%
% 個人情報
%
% 志望の動機と本人希望記入欄はlatex のコマンドを記述できます。
%
\志望の動機{
	\begin{tabular}{ll}
	{\gt 志望の動機} & 〇〇〇〇〇〇〇〇〇〇〇〇〇〇〇〇〇\\
	{\gt 特技} & 〇〇〇\\
	{\gt 好きな学科} & 〇〇〇\\
	{\gt アピールポイント} & 〇〇〇〇〇〇〇〇〇〇〇〇〇〇〇\\
	\end{tabular}
}
\本人希望記入欄{
	私が希望する仕事の条件は下記の通りです。
	\begin{itemize}
	\item 〇〇〇〇〇〇〇〇〇〇〇〇〇〇〇〇〇
	\item 〇〇〇〇〇〇〇〇〇〇〇〇〇〇〇〇〇
	\item 〇〇〇〇〇〇〇〇〇〇〇〇〇〇〇〇〇
	\end{itemize}
}

\サイン{Your Signature}

\end{document}